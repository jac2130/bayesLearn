\documentclass{article}
\usepackage{natbib}
\begin{document}
\title{} 
%\thanks[footnoteinfo]{The author acknowledges the difficulty of publishing results}
\author{Johannes Castner}
%\email{jcastner@ebay.com} 
%\affiliation{Ebay Research, San Francisco, USA}
\author{T. Maillart}
%\email{maillart@berkeley.edu}
%\affiliation{UC Berkeley, Berkeley, USA}

\begin{abstract} 
%the following has exactly 296 words and 300 are allowed. 
From finance \cite{} to the stability of democracies \cite{}, beliefs play a central role in our explanation of many phenomena \cite{}. In the social sciences, these beliefs are often conceptualized as probabilistic assessments over states of the world \cite{}.  However, these beliefs are derived from systems of coherent beliefs that people hold in their minds, regarding how the world works. This view has been supported by recent work in cognitive science \cite{lombrozo2006structure, anderson1990cognitive}.  {\bf [something is missing to articulate the 2 sentences (before and after)]}. How do humans learn in simple and in complex systems?  How efficiently do they explore the space of possible beliefs and how closely is the direction of exploration tied to experience?

Our work presents new experimental results on the rate at which people learn in more or less complex environments. We find that the rate of learning is much slower than it would be if learners were Bayesians, as had been proposed in older economic theories \citep{Boyer84, Prescott72, Rothschild74, McLennan84, Mirman84, Easley89, Kiefer89}.

 Surprisingly, we found that the learning rate is identical when people build models about more or less complex systems, although accuracy is always higher when the system is structurally simpler because initial models are better. We then propose a theoretical framework to explain the consistently slow rates of learning.  The problem seems to be linked to the well documented fact that humans are unable to generate truly random signals \citep{Jokar12}; they are pattern creators and pattern synthesizers.  Our data suggests that we are justified in putting forth the following explanation: Once human reasoners have explored regions of the space that they seek to learn something about, they tend to return to patterns that they have already explored in the past and this becomes harder and harder as time goes on and as greater parts of the space have already been explored.  Thus, learning becomes harder and slower over time. 

%The data used in this work come from an experiment that we ran using an experimental platform designed to measure how participants form and update causal beliefs in more or less complex settings.  In the experiment, participants see a data stream: data about multiple binary variables in a system, which each take on the value High or Low. For example, one variable could be a stock price and there may be a number of variables that explain its variation or are explained by its variation in turn.  A visual interface allows the subjects to draw a causal model of the system. The platform then computes predictions on the basis of the participant's causal model and makes those prediction visible to the participant. The participant then makes bets using these predictions. In doing so, the participant gets feedback on the accuracy of her beliefs, and can update her beliefs by modifying the causal model. The platform thus allows us to observe beliefs and learning in a very precise and controlled way.
\end{abstract}
%\end{frontmatter}
\bibliographystyle{apsrev4-1}
\bibliography{references}
\end{document}
