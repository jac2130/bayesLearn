\section{What we can observe from Matrices} We have produced two types of matrices, which allow to get a visual glimpse of the search process by participants. A representative example of these matrices is shown below in Figure ???. Patterns are diverse and described in more detail on the Figure and on its caption. But the triangle matric {\bf A} shows well the search process performed by the participant: High contrasted boundaries show disruptive model propositions, well homogeneous color rectangle show stability of contiguous models, perhaps with some slight iterative improvement (it's hard to see from the figure actually). Yellowish Horizontal stripes show integration of many former models, while blueish horizontal stripes, show exploration of solutions remote from all solutions proposed beforehand.  

It's unclear how we can exploit this visual insights so far, but it reminds me of some organization science theories, involving research and development versus production, where a balance is desirable, between searching for radical solutions and exploiting them for converging e.g., towards a product \cite{March_1991}. This theory certainly applies to individuals, yet the same phenomenon shall have already been documented in cognitive science {\bf [references are needed]}. 
  
  