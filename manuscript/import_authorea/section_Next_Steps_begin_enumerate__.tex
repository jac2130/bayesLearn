\section{Next Steps}

\begin{enumerate}
\item {\bf Literature review}: Scott Page stuff (his work and the work he cites), Go more in depth in the James March (Management Science), Simon, H. The science of the artificials, Ch. 4. + Cognitive Science Guys
\item Get data for number 13 (simple), and then, for all individuals
\item Exploration (resp. Exploitation) phases (dis-improvement resp. improvement dynamics): (i) inspect phases qualitatively, (ii) chilling effect. Can we verify that indeed exploration (resp. exploitation) occurs during phases of disimprovement (resp. improvement). 
\item come up with a continuous metric between exploitation and exploration.
\item at the aggregate level it looks like these phases are more or less consistent across subjects: Is there a ``universal" (self-similarity) law governing exploration versus exploitation.
\item Is the process self-similar ?
\item peaks and valley $\rightarrow$ $\Delta Perf / \Delta t = (Peak - Valley)/(t_{Peak} - t_{Valley})$, with $Valley$ the last best performance and $Peak$ the next worst performance, then $\Delta Perf / \Delta t$ as a function of $T$.
\end{enumerate}


\section{Side ideas} 

\begin{enumerate}
\item modularity: you can think of brain packaging parameters and/or sub-optimal set of parameters of interest, which are recalled in congruent/coherent way.
\item 
\item
\end{enumerate}