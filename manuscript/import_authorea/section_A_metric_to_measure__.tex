\section{A metric to measure Exploration and Expoitation}

\\

Intuitively, as people build models and adjust them, they can explore new terrain in model space or they can combine models that they have previously built. How they explore new regions or exploit regions that they have explored before can only be understood if we can measure the degree of current period exploitation and exploration.  Clearly, when people first start building probabilistic models their only choice is to explore new grounds and thus the exploration metric should be maximal at period 0.  This maximum of the exploration metric is arbitrarily set to 1 and its opposite (exploitation) is set to 0.  In fact, at every time period the exploration and exploitation metrics obey the following deterministic relationship: 

exploration$_t$ = 1-exploitation$_t$, 

with exploration$_t$, exploitation$_t$ $\in [0, 1]$. 

The exploration metric is obtained by calculating a particular distance measure, the square-root of the Jensen Shannon Divergence, between a weighted sum of all previous models and the new model. This has the property that models outside of the convex hull of all previous models are maximally explorative and models in the center of the convex hull of all previous models are maximally exploitative, as would seem intuitive.  The convex hull defined by a set of points in a two dimensional space could be thought of as that part of $\mathbb{R}^2$ that lies in the interior when a rubber band is stretched around all of the points. 