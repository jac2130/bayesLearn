\section{experiment}
The experiment conducted at Columbia University's Social Science laboratory asked 96 participants to reverse engineer a Bayesian Network with its conditional probabilities and dependencies between nodes.\\

In our experiment, participants were asked to reverse engineer a multidimensional stochastic process, explicitly expressed in the form of a Bayesian network. They were given 40 minutes, and all changes made were recorded at a 1 second resolution. Participants trying to find the best solution faced a though problem: The {\it simple} process had 3 binary variables, which means that the problem can be thought of as the simultaneous estimation of $(2^3) = 8$ parameters, or the 8 dimensional vector $\mathbf{s}$ with $0 \leqslant s_k  \leqslant 1$ for $k = \{1,...,8\}$ (resp. $k = \{1,..., 16\}$ for the 4 variable {\it complex} stochastic system).\\



\subsection{An experiment to probe cognitive mechanisms involved for solving hard problems}

%{\bf [ $\rightarrow$ this subsection needs proper citation and cognitive science lingo]}

Solutions to problems, regarding estimation of complex structures in high-dimensional joint distributions can be classified into equivalence classes, which constitute the sets of structural and parametric solutions that factor into the same joint distribution \cite{pearl2009causality, Pearl2009CMR, Koller2009PGM}.  Out of these equivalence classes, it is the objective to find the equivalence class of the data generation process which has produced the observations. Note also that if there exists a {\it causal explanation} of the process, this explanation has the simplest structure in its equivalence class \cite{Koller2009PGM} and this simplest structure is unique.  Since this process--even the structurally simplest member of its equivalence class--might be rather complex, it may be approached over time through iterative parametric refinements of a mental model of this process, as well as through sudden structural epiphanies. Yet getting close to this solution is hard because interdependencies often have unintuitive observational consequences, or rather, observations often result from  probabilistic influence that is unintuitive to humans. Human intuitive failure with respect to probabilistic influence and its logical consequences is best illustrated through the famous Monty Hall problem \cite{Blackburn08Phil, Honderich05Phil, Upton14Statistics, Colman08Psych}. \\

People tackling hard problems face a tension between testing and updating their beliefs from parameters and structures stored in their memory (i.e., {\it exploitation} or {\it recombination of mental structures}), and taking action to {\it explore} and update their beliefs from not previously available mental structure. Taking such action is cognitively equivalent to the exploration of unknown territories by pioneers in the physical world, or more directly, to scientists asking and then testing new hypotheses. The former {\it exploitation} approach may bring improvement toward the solution but it is limited to a {\bf convex combination} of previously tried solutions. The latter approach carries higher potential risks (behind the hill a leopard might be lurking, or research funds might be wasted on finding nothing of interest) as well as higher potential returns (there might be an unmeasurable treasure hidden behind the hill, or a cure for cancer might be found), but whether exploration brings improvement towards the solution at some given moment or not, this strategy expands the cognitive frontier. Below we refer to the convex hull of previously explored solutions as the {\it cognitive frontier}.  Once a new portion of the solution space has been explored, the attempted proposed model is then stored into memory and may be recombined, later on, with other proposed models, in proportion with its believed usefulness. \\
