\section{Discussion}
We find that cognitive performance ($1-$ distance achieved at time $t$) is associated with the number of distinct sites visited over time. For that, three properties are critical : First, the number of proposed solutions divided by time $t$, which is the rate of proposed solutions.  Second, the propensity to return to previously visited solutions due to memory. And finally, third, the ability to combine previously visited solutions with new, explorative ones, through an exploration vs. exploitation process.  The class of statistical models that has been used to describe such processes is known as self-exciting Hawkes processes. 

Some empirical puzzles and theoretical directions remain to be explored. 
In our data we document that with time, the Jensen–Shannon divergence between the model and the data generating process decreases, which is to say that our participants' solutions improve.  Improvement can be due to (i) luck and chance, which we consider unlikely, or due to (ii) convergence toward the true model through learning.  Learning is in turn slowed due to memory related stickiness. 
  
We connect the cascades observed in our data to the Hawkes processes which describe them well. 

The Hawkes processes relate to evolutionary cognitive science theories that explain the patterns of increased and decreased success through cascades in our data and the human memory giving rise to those patterns.  What are the proximate and ultimate causes that move our cognitive and behavioral patterns to follow these strange processes? To understand these processes, we will examine the time periods between return visits to past solutions.  When returns are clustered closely in time then they have much less of an effect on performance than if they are more spread out over time.  


%- connect results with Distance Decay, but basically a model could be summarized as $Distance \sim S_T$ (by the way $D_{min}$ versus $S_T$ could be improved by looking at $D_t$ versus $S_t$). $S_T$ (resp. $S_t$) is a function of displacement $\Delta r$ decisions (which may also cost additional time) and (obviously) their influence on score. 

In summary, at different points in time, a participant must make decisions about displacement, with a part of ``risk", which may be discounted by time spent (waiting time relative to avg waiting time and/or experimentation duration and/or time left). Large displacements may be associated with riskier exploration and more time required for decisions. There exists a $\Delta r$ which maximizes improvement ($\Delta r \approx 0.15$). Such move size maximizes the chance that a new point in the abstract space will be visited.



{\bf [cognitive load]}









