\section{Discussion}

Outstanding problems :

- decreasing mean square displacement: it basically seems that with time displacement decreases: This can be due to (i) the limited space (unlikely), (ii) some convergence toward the true model, or (iii) some stickiness, or (iv) probably (ii) and (iii) together.
  
- connecting cascades and memory with (i) evolutionary theory and (ii) increased success (resp. counter-performance). Connect also with potential cascading processes in cognition (any knowledge about this? memory?)

- time between return visits (if return visits happen in close time, then it matters little)

- connect results with Distance Decay, but basically a model could be summarized as $Distance \sim S_T$ (by the way $D_{min}$ versus $S_T$ could be improved by looking at $D_t$ versus $S_t$). $S_T$ (resp. $S_t$) is a function of displacement $\Delta r$ decisions (which may also cost additional time) and (obviously) their influence on score. 

In summary, there is at some point a displacement decision with a part of ``risk", which can by the way be discounted by time spent (waiting time relative to avg waiting time and/or experimentation duration and/or time left). Large displacement can be associated with more risky exploration and more time required for decision. It looks like there is a $\Delta r$ which maximize improvement ($\Delta r \approx 0.15$). This is interesting because it suggests that such move size maximizes the chance that a new patch will be visited.


potential long reach :

- qualitative/quantitative differences between performing/non-performing participants?












