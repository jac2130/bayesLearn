\section{Discussion}
We find that performance (min distance achieved) is associated with the number of distinct sites visited over time. For that, three aspects are critical : First, the number of proposed solutions (translates into the rate of proposed solutions),  Second, the propensity to return to previously visited solutions, and finally, third, the ability to combine previously visited solutions with new, explorative ones, through an exploration vs. exploitation process, combined with a memory function.  The class of statistical models that has been used to describe such process is known as self-exciting Hawkes processes. 

Some empirical puzzles and theoretical directions remain to be explored. 
In our data we document that with time, mean square displacement decreases, which is another way to say that our participants' solutions improve: This can be due to (i) space limitations, so that their models improve only by chance, which we consider unlikely, (ii) convergence toward the true model, or (iii) stickiness, or, most likely (iv), (ii) and (iii) together.
  
Further, we seek to connect the cascades, described by our data and by the Hawkes processes which describe our data and the human memory giving rise to those data with evolutionary theories and cognitive science about the links between cascades and increased success (resp. counter-performance).  What are the proximate and ultimate causes that move our cognitive and behavioral patterns to follow these strange processes? Next, we will examine the time periods between return visits to past solutions.  If returns are clustered closely in time then they would have much less of an effect on performance than if they are more spread out over time.  


%- connect results with Distance Decay, but basically a model could be summarized as $Distance \sim S_T$ (by the way $D_{min}$ versus $S_T$ could be improved by looking at $D_t$ versus $S_t$). $S_T$ (resp. $S_t$) is a function of displacement $\Delta r$ decisions (which may also cost additional time) and (obviously) their influence on score. 

In summary, at different points in time, a participant must make decisions about displacement, with a part of ``risk", which may be discounted by time spent (waiting time relative to avg waiting time and/or experimentation duration and/or time left). Large displacements may be associated with riskier exploration and more time required for decisions. There exists a $\Delta r$ which maximizes improvement ($\Delta r \approx 0.15$). This suggests that such move size maximizes the chance that a new point in the abstract space will be visited.













