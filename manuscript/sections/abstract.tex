
Recent research \cite{baronchelli2013levy} has suggested that cognitive mental search patterns \cite{rhodes2007human,radicchi2012rationality,radicchi2012evolution} may be inherited 
from typical foraging and mobility patterns of animals \cite{viswanathan1996levy,ramos2004levy,reynolds2007displaced} and humans \cite{gonzalez2008understanding,song2010modelling,rhee2011levy}, similarly to 
hunter-gatherers L\'evy flight search algorithms \cite{brown2007levy,raichlen2014evidence}. Here, we study the mental search trajectories of  
individuals who have been asked to reverse engineer the conditional probabilities of 3-node and 4-node Bayesian 
networks (48 participants per treatment) \cite{steyvers2003inferring,pearl2009causality}. We find that, in contrast with the random 
trajectories predicted by the L\'evy flight and random walk models optimized for search of sparse targets \cite{viswanathan1999optimizing,edwards2007revisiting,song2010modelling,viswanathan2011physics}, human mental 
search exhibits temporal and spatial regularity, characterized by a time dependent distance 
between two consecutive solutions proposed by each individual, as well as a tendency 
to return to previously tested solutions. Both contribute to a sub-optimal exploration of the 
problem space, which in turn limits the odds of getting close to the target solution. Our results 
suggest that some cognitive limitations associated with exploring complex abstract spaces are 
deeply hard-wired in our brain, and may stem from previously evolutionary fit hunting strategies, 
which reveal largely inefficient for solving problems with modern complexity.
