Research in cognitive science has recently turned its attention to probing the human {\it capacity to infer causal structures from both observation and intervention, and to choose informative interventions on the basis of observational data} \cite{steyvers2003inferring,pearl2009causality}. Here, we report the fine-grained mental search trajectories of 96 individuals incentivized to reverse engineer the joint probability distributions of 3- and 4-node Bayesian networks (48 participants per treatment). We find that individuals adopt a ballistic Continuous Time Random Walk (CTRW) search process \cite{}, which has been found to be optimal for search of sparse solutions in large spaces \cite{viswanathan_optimizing_1999,edwards_revisiting_2007,song_modelling_2010,viswanathan_physics_2011}\footnote{With respect to search behavior, we find no significant differences between treatments.}. Yet the observed search process for humans inferring causal structures is not memoryless. It rather involves several memory processes, including recombination of formerly visited solutions, and repeated return to previously visited solutions. Here, we characterize how fine-grained moves made by individuals may improve the prevailing mental model, or how sometimes on the contrary, some moves degrade this model. We find that {\bf XXXX}... {\bf [a description of explore / exploit is missing]}. In nature and society, ballistic search processes have been found to describe behavior of animals \cite{baronchelli_levy_2013} and humans \cite{gonzalez_understanding_2008,song_modelling_2010,rhee_levy-walk_2011} involved in food search patterns in case of sparse food spots over large territories. Similar search patterns have been found for search in abstract spaces \cite{rhodes_human_2007,radicchi_rationality_2012,radicchi_evolution_2012}, prompting suggestion that search strategies used for abstract search \cite{baronchelli_levy_2013} may be inherited from foraging and mobility strategies by hunter-gatherers \cite{brown_levy_2007,raichlen_evidence_2014}. We use direct evidence from a cognitive experiment conducted in a social science laboratory, to further discuss this proposition {\bf [to be refined when we converge]}. {\it Our results, then suggest that cognitive limitations associated with exploring complex abstract spaces are deeply hard-wired in our brain and may stem from previously evolutionary fit search strategies.  Human cognition, then, maybe grossly inefficient for solving problems with modern complexity.}

