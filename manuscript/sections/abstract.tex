Recent research \cite{baronchelli2013levy} has suggested that cognitive search patterns through abstract spaces \cite{rhodes2007human,radicchi2012rationality,radicchi2012evolution} may be inherited from foraging and mobility patterns of animals \cite{viswanathan1996levy,ramos2004levy,reynolds2007displaced} and early humans \cite{gonzalez2008understanding,song2010modelling,rhee2011levy}. The patterns are similar to those that are observed when L\'evy flight search algorithms are employed, as is the case when hunter-gatherers search for resources in some geographic terrain \cite{brown2007levy,raichlen2014evidence}. 

Here, we study the mental search trajectories of participants in our cognitive science experiment who have been asked to reverse engineer the joint-probability distributions of 3-node and 4-node Bayesian networks (48 participants per treatment) \cite{steyvers2003inferring,pearl2009causality}. We find that--contrasting with the trajectories predicted by L\'evy flight models and random walk models that are optimized for search of sparse targets \cite{viswanathan1999optimizing,edwards2007revisiting,song2010modelling,viswanathan2011physics}--the cognitive search of our participants exhibits temporal and spatial regularity. This regularity is characterized by a time dependent distance between two consecutive solutions that are proposed by each individual.  Additionally, we observe a tendency to return to previously tested solutions. 

Both types of search processes are sub-optimal with regards to exploration of the comparatively complex problem spaces. When they are employed, these search strategies lead to much slower than optimal convergence towards the target distribution. Our results, then suggest that cognitive limitations associated with exploring complex abstract spaces are deeply hard-wired in our brain and may stem from previously evolutionary fit search strategies.  Human cognition then, maybe grossly inefficient for solving problems with modern complexity.
