\section{Introduction}
When searching for sparse (resp. unique) solutions in complex problem spaces, animals and humans resort to L\'evy walks/flights random walk search strategies, which alternate between local (short range) and long-range moves. The optimal search strategy in this kind of situation involves alternating  clusters of many local moves with punctual long range displacements \cite{viswanathan1999optimizing, edwards2007revisiting,song2010modelling,viswanathan2011physics}. Number of species use this strategy to search for food in wide areas, and it has been shown that hunter-gatherers behave similarly, a well as urban people as traced from their banknotes \cite{brockmann2006scaling} or mobile phone traces \cite{gonzalez2008understanding,song2010modelling}.\\

The cognitive  as well as the evolutionary \cite{radicchi2012evolution} underpinnings of L\'evy walks by humans has been questioned and investigated. The ramifications of L\'evy walks in the mind with foraging / mobility patterns in the physical space: here we question if 
%our mind has been shaped by evolutionary pressure and 
humans resort to similar strategies. Recent research on online bids \cite{radicchi2012rationality} found similar L\'evy walk patterns even though they appear sup-optimal and even slightly irrational, hence suggesting that L\'evy walks are somehow hard coded in our mind  \cite{radicchi2012evolution}, as humans resort to search strategies, which are no longer fit in the world of information and reasoning with abstract problems, which resolution brings its own incentives and resource rewards in the form of money, recognition, reputation, and pleasure \cite{rewards_modern_societies}.\\

Here, considering humans facing a hard problem -- typically one a solution in a complicated problem mental search patterns \cite{}�may resemble  \cite{baronchelli2013levy} has suggested that cognitive mental search patterns \cite{rhodes2007human,radicchi2012rationality,radicchi2012evolution} may be inherited from similar L\'evy walks/flights foraging and mobility patterns.\\

However, these findings were based on specific cases, involving a specific type of auctions \cite{}. It is also considered that the strategy used is L\'evy walks/flights because it is optimal. However, only truly random L\'evy walks/flights are optimal. True randomness is equivalent to a memoryless process, which would allow exploring the solution space with no consideration of previous knowledge. (a evolutionary theory framework is used to rationalize their findings).\\

Assuming that people follow a memory-less process is problematic, because clearly people resort to all kinds of short and long term memory processes, which allow them to explore new potential solutions, and to combine these explorations with past experience {\bf [a truly cognitive science paragraph is needed here]}. \footnote{One instance of memory is return to previous site (resp. previous solutions), evidence from mobility shows -- not surprisingly -- that people tend to return to previously visited sites \cite{}.}\\

Searching for (or tending/optimizing to) a unique solution involves try and fail, and progressive learning. One may think of an evolutionary process (e.g., ``Animals explore the environment mainly for searching food resources, and it is therefore plausible to ascribe the optimality of their
search strategies to a selective evolutionary process."  \cite{radicchi2012evolution}), a Markov process \cite{}, or a process with long range memory, in which candidate solutions explored in the past are reused, and recombined with more recent explored solutions. Finally, some new solutions are truly explored out of the currently explored solution envelop.\\

Here, we have conducted an experiment in which participants reverse-engineered the influence structure of two kinds of simple Bayesian networks (made of resp. 3 and 4 nodes). We find that humans search strategies alternate local search patterns with long range relocations (i.e., {\it flights}) over the solution space. Yet they tend to return preferentially to already visited locations.\\

However, and on the contrary to former theories (say which ones), the search path is not memoryless: humans tend to return to previously visited locations. This {\it stickiness} limits the capacity to explore the broader solution space in order to get closer to the solution. As such this behavior may be considered as a genuine cognitive bias limiting the convergence to the correct solution, which indeed exhibits a particularly slow decay (see Figure \ref{fig:vs_dr}A)  However, from an evolutionary perspective, sticking to some places makes sense. A good hunting spot is a place where there is food renewal, which creates incentives to regularly return to check out for newly available prey: Typically, a lion may return regularly to a river bank where zebras get fresh water frequently. \\

 {\bf [slow convergence we find that the decay of the distance to the true model follows a power law $\sim t^{-\nu}$ with $\nu \approx 0.15(1)$]}
 
 \subsection{Experiment}
In our experiment, participants were asked to reverse engineer a Bayesian network. They were given 40 minutes, and all changes made were recorded at a 1 second resolution. Participants trying to reverse the best solution face a though problem: The {\it simple} Bayesian network has 3 nodes, and is defined by a 8-parameters vector $\mathbf{s}$ with $0 \leqslant s_k  \leqslant 1$ for $k = \{1,...,8\}$ (resp. $k = \{1,..., 16\}$ for the 4 node {\it complex} Bayesian network). 
 
 
\subsection{L\'evy Flight / CTRW}
We first find that the search process follows a L\'evy flight process with waiting times between moves are random variables, which can be accounted together as a continuous time random walk. Both the distributions of displacement (Figure \ref{fig:pdfs}A ) and waiting times (Figure \ref{fig:pdfs}B ) exhibit power law distributions  (Probability density function of displacement $pdf(\Delta r) = \Delta r^{-\alpha -1}$ with $\alpha = 0.40(5)$. {\bf B.} Probability density function of waiting time $\Delta t$ $pdf(\Delta t) = \Delta t^{-\beta -1}$ with 2 regimes : $\beta_{\Delta t < 125} = 0.38(4)$ and $\beta_{\Delta t > 125} = 1.59(5)$. Distributions of $\Delta r$ and $\Delta t$ are equivalent for the simple and complex treatments.)

{\bf Problem :} Such CTRW process should normally follow ballistic diffusion characterized by mean square displacement (MSD) and diffusion $\sim t^{\mu}$ with  $\mu_{Levy} = 1$ or super-diffusion $\mu_{CTRW} = \beta$ \cite{21,23}). Here, however, mean square displacement (MSD) decays as $\sim t^{\mu}$ with $\mu_{simple} =-0.23(2)$ and $\mu_{complex} =- 0.26(1)$ showing a slow convergence. 



 
 %Here, we document how such frustrating and somewhat irrational situations may stem from evolutionary homology \cite{evolutionary_homology}, that is mental search properties may share conserved neural substrates with similar neuro-molecular processes guiding spatial search in animals and modulating the control of human attention \cite{hills2006animal}. We show that search strategies inherited from food \cite{food_foraging} and resource \cite{resource_foraging} foraging lead to a form of ``hard-wired bounded rationality" when tackling hard problems.\\

%This behavior is at odds with L\'evy walks food search strategies 
%\cite{iswanathan2011physics} used by number of animals 
%\cite{viswanathan1996levy,reynolds2007displaced,edwards2007revisiting,ramos2004levy} and hunter gatherers 
%\cite{brown2007levy}. L\'evy walks which displacement obeys a power law distribution 
%$P(R > \Delta r) \sim \Delta r^{\alpha}$ with $ 0 < \alpha \leqslant 2 $ are known maximize 
%displacement through a super-diffusive process, while minimizing the probability to return to an 
%already visited site (on the contrary to Brownian motion \cite{humphries2010environmental,mendez2013stochastic}). In 
%particular, the food search process in nature has also been found to be optimal $\alpha \approx 1$ 
%\cite{viswanathan1999optimizing}, in the case of Hadza hunter-gatherers in Tanzania 
%\cite{raichlen2014evidence}, and for the dissemination of mussels \cite{de2011levy}. In the latter 
%case, the optimal search process stems from the cooperative organization of mussels, and how 
%this cooperation shapes the environment \cite{de2011levy}. Human mobility traced by banknote circulation \cite{brockmann2006scaling} or mobile phone tracking \cite{song2010modelling,rhee2011levy} also exhibits highly regular L\'evy walk patterns \cite{gonzalez2008understanding}.

%There is suggestive evidence that some cognitive mechanisms, such as information/memory retrieval follows a L\'evy walk \cite{hills2012optimal} with some limited analogy with optimal foraging \cite{rhodes2007human} {\bf (in \cite{rhodes2007human}, it's unclear why the waiting times should be optimal with $\mu \rightarrow 1$. The optimality is in space, optimality is less clear in time)}. {\bf (random walks \cite{abbott2015random})}

%For some time now, cognitive scientists have posed that modeling people's beliefs about causation in the material world provides an effective explanatory framework for how we process information and learn from these observations \cite{Griffiths2008}.  Probabilistic causes and effects, in other words, make up an important construct we seem to make use of when we seek to understand our world.  Bayes Nets are well-defined mathematical objects corresponding to an intuitive representation of causal beliefs.  They are easy to manipulate, do inference with and use as a means to compare different beliefs.  The kinds of Bayes Nets that have been allowed for in experimental settings up to this point, however, have been extremely simplistic and there has been an unanswered call for learning experiments with increased complexity (Griffiths?).  Our work is to our knowledge the first answer to this call.

%From financial stability to the stability of democracies, beliefs play a central role in our explanation of many phenomena. In the social sciences, these beliefs are often conceptualized as probabilistic assessments over states of the world.  However, they are derived from coherent belief systems people hold in their minds regarding how the world works as supported by recent work in cognitive science \cite{lombrozo2006structure, anderson1990cognitive}.  How do humans learn in simple and in complex systems?  How efficiently do they explore the space of possible beliefs and how closely is the direction of exploration tied to experience?  Bayesian approaches of learning would not be feasible for learning in realistically complex systems, given recall constraints and they are not defined in cases where initial beliefs are such that they put zero weight on some possible outcomes.

%Our work presents new experimental results on the rate at which people learn in more or less complex environments. We find that learning rates are much slower than they would be if learners were Bayesians, as had been proposed in older economic theories \cite{Boyer84, Prescott72, Rothschild74, McLennan84, Mirman84, Easley89, Kiefer89}. We also find that the learning rate is slightly higher when people build models of systems that are less complex and that even if the rates were identical accross levels of complexity, accuracy is always higher when the system is structurally simpler because initial models are closer aligned with reality.

%When searching for solutions to outstanding problems, in this case the formation of coherant belief structures that explain experience, humans must come up with innovative solutions.  We show that their strategies look a lot like a class of random search processes in which complete random guessing is strategically combined with the consolidation of past and current experience. We show how people shape their beliefs, starting with complete guesswork through a process that can be accurately modeled as a L'evy random search process, which involves both synthesis of current knowledge (mental exploitation) and out-of-the box mental exploration (see Figure \ref{fig:schematic}). Such random search processes are ubiquitous across the life sciences, in cognitive science, computer science  and artificial intelligence. We then measure how this process leads to convergence, albeit slow, to the correct stochastic solution.

%As can be seen in Figure \ref{fig:decay}, the distance between a person's evolving belief system about some physical system and the long-run absorbing distribution, which we interpret as a performance score, decays with time in a very specific way:
%
%\begin{equation}
%\label{ }
%d(belief_{i, t}, absorbing distribution) = \alpha*t^{-\beta_i}
%\end{equation}
%
%The functional form of human learning in the context of our experiment, then, is the same as for the Bayesian learner, but the decay rate, $\beta$, of a human learner is much smaller than that of a Bayesian.
