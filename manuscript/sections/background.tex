\section{Background}
Searching and finding solutions bring rewards, some time for their own sake \cite{hedonism}, but more often for survival \cite{maynardsmith1978}, which implies finding food directly (foraging) or indirectly (through monetary means). To strive, food search shall be economic, i.e., the search cost shall not be larger than the energetic intake \cite{pyke1984} and optimization shall occur through either ``time minimization" (time spent feeding leaves more time for other activities) or ``energy maximization" (fixed time in which to feed during which it aims to maximize its energy gain) \cite{Schoener (87)  in maynardsmith1978}.\\

The evolution of foraging behavior is subject to functional (e.g., morphologies and physical properties \cite{Pyke1984}) and environmental constraints (e.g., distribution and accessibility of food patches on complex environmental landscapes \cite{}). The capacity to sense, store and process foraging information is critical \cite{125 , 189, 268 in Pyke 1984}. Animals may obtain information through either direct experience, observation of others \cite{ref274inPyke1984}, or through basic or advanced collective intelligence mechanisms \cite{}. Foraging is also assume to evolve more rapidly than the rate at which relevant conditions change \cite{ref210inPyke1984}. In that sense, animals are expected to adapt and optimize their foraging \cite{Pyke1984}.\\

%{\it If the animal is alone in a uniform environment, no difficulty arises.
%But if we allow for competition and for a changing environment, several
%choices of optimization procedure are possible. For example, three 
%possibilities arise if we allow just for competition:} $\rightarrow$ capacity 
%to anticipate, build relevant scenarios </Maynard1984> 

Foraging in natural environment most often implies to find sparsely distributed food sites. Levandowsky et al. \cite{} have suggested that microorganisms may search for sites following a Levy flight process, defined as 

\be
eq~Levyflight
\ee 

A Levy distribution is advantageous  when target sites are sparsely and randomly distributed, because the probability of returning to a previously visited site is smaller than for a Gaussian distribution. The Levy flight may also help overcome the problem of multiple animals searching food at the same time \cite{Shlesinger6, viswanathan1996}. Besides, microorganisms, potential evidence for Levy flight search patterns have been identified for animals such as albatrosses \cite{}, bumblebees \cite{}, deers \cite{}, marine predators \cite{sims2008}, spider monkeys \cite{Ramos-Fern�ndez} and for humans Dobe Ju/�hoansi hunter gatherers  \cite{Brown2007}. \footnote{Dobe Ju/�hoansi foraging hunters and foragers living in and around the Kalahari Desert in Botswana and Namibia. They have been
intensively studied with special attention to their subsistence
system and economy (Lee, 1979, 1993; Lee and
DeVore, 1976).} \\

Additional considerations were proposed regarding optimization of the Levy flight process, and it was shown that $\mu = 2$ is an optimal Levy flight search pattern \cite{Viswanathan1999}. (add other references) \\

The L\'evy flight search process has enjoyed much attention for its simplicity. Yet, more careful attention to existing data and increased sensing capacities show that Levy flight may not hold at all for some animals such as albatrosses, bees, bumblebees and deer \cite{reynolds2007}, may be more complicated \cite{}. In addition, it is debatable if animals and humans perform search according to a general mechanism that would be hard coded, or if they may switch regime based on strategic choices and on information they can gather. If for {\bf primitive} animals and micoorganisms, a pure random search mechanims (i.e., without integration of information) may be assumed, more advanced {\bf ``animals"} can certainly integrate and memorize more information, which in turn is critical to optimize search. Such considerations has led researchers to propose more advanced models, which stem from theoretical considerations and somewhat fit the better. {\bf [add the fancy model from Royal Society Interfaces \cite{Zhao2015}]}. Such models include the Levy-modulated correlated random walks (LMCRWs), which implies a correlated directional component (i.e., the tendency to search in a more or loss linear way, without making random turns) \cite{Bartumeus2005}. Plank and James (2008) proposed that 2 brownian motions with regime switching between intensive and extensive search, may be indistinguishiblabe from a Levy flight from the data, and appears to be a more efficient search strategy in the case of non-destructive foraging�\cite{plank2008}. This results highlights that simple decision rules, which arguably cost little brain power such as switching between a couple of strategies, may significantly optimize search.\\

%We study the territory covered by N Levy flights by calculating the mean number of distinct sites, $^SN(n)$, visited after n time steps on a d-dimensional, $d>2$, lattice \cite{Berkolaiko1997}
\subsection{Modern human mobility \& memory}
Human mobility in modern times is less a story of foraging than a question of understanding mobility in everyday's life, which may largely devoted to gathering resources (e.g., getting to school, to work, socially interact and exchange knowledge, or for mating \cite{sexual contacts}) . While early studies have relied on banknote circulation \cite{brockmann} to account for mobility, the mobility question has been greatly facilitated with cellphone data \cite{gonzales}. Most datasets exhibit Levy flight patterns suggesting a universal law of foraging, search and mobility \cite{}.\\

It (obviously) appears that mobility patterns are not completely random. On the contrary, they encompass a significant amount of memory (people tend to live at the same place, work or study at the same place, meet socially at the same location). Song et al. proposed a mechanism inspired from proportional growth \cite{} (a.k.a. preferential attachment \cite{}) and incorporated it into a continuous time random walk model \cite{}, which also accounts for waiting times between displacements, which happen to be also most often heavy-tailed \cite{} and could be explained by task prioritization \cite{}, periodicity such as circadian rythms \cite{} (on the contrary to travel times for animals or humans walking over large distances ?). Yet, proportional growth appears to be unsuited, or at best an incomplete model. Indeed, proportional growth prescribes that the rate at which people return to a site should grow exponentially, which appears to be counter intuitive (e.g. people stay at their home mostly at the constant rate, arriving in the evening and leaving in the morning). The other problem, which has been pointed out \cite{barbossa,szell} is recency: in human mobility individuals tend to return more often to previously visited sites, with a memory function (which remains to be characterized).\\

%$\rightarrow$ In the physical space, there is usually an implicit %dependence between distance and time, which has by the way led to %issues regarding the validty of Levy walks \cite{Viswanathan2007} 

%<Ramos-Fernandez 2004>
%First, the length of a trajectory�s constituent steps,
%or continuous moves in the same direction, is best
%described by a power-law distribution in which the
%frequency of ever larger steps decreases as a negative
%power function of their length. The rate of this decrease is
%very close to that predicted by a previous analytical Levy walk model to be an optimal strategy to search for scarce
%resources distributed at random. Second, the frequency
%distribution of the duration of stops or waiting times also
%approximates to a power-law function.
%</Ramos-Fernandez>

\subsection{Cognitive mechanisms involved in mental search and optimization}
There are at least two reasons to consider that cognitive mechanisms involving mental search may as well exhibit L\'evy flight patterns (yet not necessarily random search as shown above). \\

First, it may be an evolutionary trait inherited from former hunter-gatherer food search strategies, such as we have retained many other traits for this period of human development, such as .... \cite{}\\

Second, humans spend a considerable amount of their time searching for solutions to optimize the way they gather their most needed resources at broad (i.e., food, financial resources, pleasure, social ties, etc). Moreover, the capacity to find solutions, which are not obvious to others, may bring a substantial competitive advantage (e.g., a trader whose job is to find and implement arbitrage strategies, an entrepreneur who have found a readily implementable solution to an outstanding problem). For that, they presumably consume a substantial amount of their energy resources to tasks involving cognition \cite{}, and they may also rely on chance, which is a hallmark of high risk high return strategies \cite{}.\\

Evidence suggesting that Levy-flight search patterns occur human cognition exists. Rhodes et al. \cite{rhodes2007} performed a simple experiment involving memory retrieval. They found that retrieval occurs in bursts and these bursts often involve clusters of closely related words \cite{ref4inRhodes2007}. {\bf [more details needed here. Read again the paper.]}\\

Radicchi et al. \cite{radicchiPloSOne} found that individual betting in lowest unique bid online auctions (the bidder with the lowest unique bid wins) , adopt L\'evy flight search strategies in their exploration of  ``bid space". Unique bid online auctions prescribe that the winner should find a solution, which has not been found by any other fellow bidder (here, a justification of large excursions shall be more justified by the search of unicity equiv. to searching a food site that no one else has found yet [$\rightarrow$ rephrase and c.f., literature for that]. In the special case of lowest unique bid, they designed a model, which shows that individuals have interest to bid with the optimal strategy (power law exponent $\gamma \simeq 1.27$) only if all other players play rationally. Radicchi et al. \cite{radicchiPRE} extended their research and built a model for power law exponent optimization, based on a Moran process ({\bf ``at the end of each game, the winner of the auction generates an offspring to which her/his search exponent $\alpha$ is transmitted. The new individual enters the population endowed with an exponent $\alpha + \xi$ (with $\xi$ random mutation), while a randomly extracted individual is removed in order to maintain the population size constant"}). They found that indeed the L\'evy flight power law exponent is optimized towards value around 1.5. Thus, from animal and human mobility for food search to human memory to most contemporary online auction systems, L\'evy flights ermerge as a unifying concept across generations and thus, suggests that evolutionary old parts of the brain may be involved in resource retrieval and search processes \cite{Baronchelli2013}.-

\subsection{Research on Mental search in cognitive science}


{\bf [Johannes $\rightarrow$ please build up on the background section I just wrote, and please make a short AND comprehensive of the background literature (200 words max), which made you study reverse engineering Bayesian network ]}



%Sampling with replacement is limited,
%however, by the fact that it predicts a smooth curve and is thus not well suited to accommodating the key
%observations that successive retrievals tend to occur intermittently in bursts (Fig. 1a) and that the bursts often
%involve clusters (e.g., chinchilla, skunk, mink, beaver; from Ref. [4]).
%
%{\it At an abstract
%ARTICLE IN PRESS
%Fig. 1. (a) The IRI in seconds between successive retrievals plotted as a function of position in the sequence of IRIs for two participants (3
%and 4 in Table 1). (b) The IRI sequences of (a) are plotted in the standard format as the negatively accelerated growth of the cumulative
%recall over the time course of recall. (c) The IRI sequences of (a) are plotted following subtraction of the exponential growth depicted in
%(b). 
%
%At an abstract dynamical level, foraging for foods of a particular type and searching for words of a particular type must be processes of like kind if particular foods and particular words are randomly and sparsely located in their respective spaces (niche, memory) at sites that are not previously known.}

%{\bf [ $\rightarrow$ this subsection needs proper citation and cognitive science lingo]}

{\bf [I don't understand this paragraph. It needs much clearer explanation for the lay person (your mother should be able to understand), and I suppose much more references if we want to keep it in the background section, which is where I believe it should be.]}Solutions to problems, regarding estimation of complex structures in high-dimensional joint distributions can be classified into equivalence classes, which constitute the sets of structural and parametric solutions that factor into the same joint distribution \cite{pearl2009causality, Pearl2009CMR, Koller2009PGM}.  Out of these equivalence classes, it is the objective to find the equivalence class of the data generation process which has produced the observations. Note also that if there exists a {\it causal explanation} of the process, this explanation has the simplest structure in its equivalence class \cite{Koller2009PGM} and this simplest structure is unique.  Since this process--even the structurally simplest member of its equivalence class--might be rather complex, it may be approached over time through iterative parametric refinements of a mental model of this process, as well as through sudden structural epiphanies. Yet getting close to this solution is hard because interdependencies often have unintuitive observational consequences, or rather, observations often result from  probabilistic influence that is unintuitive to humans. Human intuitive failure with respect to probabilistic influence and its logical consequences is best illustrated through the famous Monty Hall problem \cite{Blackburn08Phil, Honderich05Phil, Upton14Statistics, Colman08Psych}. \\

{\bf [This paragraph is great but it should be rewritten as a background paragraph, i.e., by engaging the literature, as much as possible]}. People tackling hard problems face a tension between testing and updating their beliefs from parameters and structures stored in their memory (i.e., {\it exploitation} or {\it recombination of mental structures}), and taking action to {\it explore} and update their beliefs from not previously available mental structure. Taking such action is cognitively equivalent to the exploration of unknown territories by pioneers in the physical world, or more directly, to scientists asking and then testing new hypotheses. The former {\it exploitation} approach may bring improvement toward the solution but it is limited to a {\bf convex combination} of previously tried solutions. The latter approach carries higher potential risks (behind the hill a leopard might be lurking, or research funds might be wasted on finding nothing of interest) as well as higher potential returns (there might be an unmeasurable treasure hidden behind the hill, or a cure for cancer might be found), but whether exploration brings improvement towards the solution at some given moment or not, this strategy expands the cognitive frontier. Below we refer to the convex hull of previously explored solutions as the {\it cognitive frontier}.  Once a new portion of the solution space has been explored, the attempted proposed model is then stored into memory and may be recombined, later on, with other proposed models, in proportion with its believed usefulness. \\

{\bf [This paragraph was commented out. Even though I know little of this, I thought it could be given a chance to be rewritten as part of the background section. If you think there is no value, please remove altogether.]} \textcolor{red}{
On average, especially when incentives are higher or tasks simple, subjects update their beliefs based on new data in a way that is consistent with Bayes rule.  This is demonstrated in numerous papers \cite{Griffiths2008}EXPLAIN + CITE EXPERIMENTS
However, at the individual level, there are large variations, as well as systematic departures under certain conditions. 
1) observed deviations from Bayesian learning
this can be organized by 
- "biases" in the priors (i.e. hypotheses), in particular deterministic bias
- deviations in the updating process: 
* representativeness heuristic/availability bias : tendency to overweight the strength of an observation and underweight its weight. Griffin and Tversky 1992 \cite{griffin1992weighing}; Holt and Smith 2009 \cite{holt2009update} ; Grether 1992 \cite{grether1992testing}. This is connected with the neglect of base rates (subjects overweigh the likelihood relative to the base rates/objective priors), and the resulting « law of small numbers » (over-generalizing from small number of observations), see Rabin 2002 \cite{rabin2002perspective}.
* ambiguous information is taken to be confirmation of current hypothesis, which is an obstacle to learning
- non-optimal acquisition of information/memory: recency bias, search for confirming evidence rather than disconfirmatory evidence (this last category is in my opinion not that relevant for your case) prediction: people have difficulties performing contingent reasoning on future events (Charness and Levin 2009 \cite{charness2009origin}). Additionally, Bayes rules makes no prediction about how learners should react to zero probability events, nor how learners should structure the hypothesis space when there are no objectively known base rates of events ad nlearners have to learn about the whole structure of the underlying environment, not just a few parameters (basically, in most natural environments, such as in markets). This is the situation explored by Ortoleva \cite{ortoleva2012modeling} and that is rarely examined in experimental settings  These departures are clues to the heuristics humans use in their judgments, which overall often approximate bayesian inference but are not equivalent. Although many of these deviations imply that learning should be slower than predicted by Bayes rule,  few experiments measure the rate of learning and how far it departs from Bayesian inference. We find that it does (slower). Furthermore, we also find that the quality of the inferences at each time space and across players varies a lot, following a "punctuated equilibrium" pattern. These observations are thus additional clues about the cognitive processes at play, with fine-grained information about how subjects update their beliefs in response to new information and the success and failure of their bets. Below were review some of the promising cognitive mechanisms posited to explain some of the deviations above and which can be evaluated in light of our data. 2) Cognitive mechanisms  Interesting mechanisms have to do with how people organize the hypothesis space and sample from it.  - sampling hypothesis - change of paradigm when observing "surprising events" (Ortoleva) \cite{ortoleva2012modeling} the satisficing principle may also mediate the learning process because computations such as Bayes rules are costly.
}


