The cognitive science literature has found much evidence in support of consistency with as well as deviance from human learning with Bayesian constructs of learning. 

On average, especially when incentives are higher or tasks simple, subjects update their beliefs based on new data in a way that is consistent with Bayes rule. EXPLAIN + CITE EXPERIMENTS
However, at the individual level, there are large variations, as well as systematic departures under certain conditions. 
1) observed deviations from Bayesian learning
this can be organized by 
- "biases" in the priors (i.e. hypotheses), in particular deterministic bias
- deviations in the updating process: 
* representativeness heuristic/availability bias : tendency to overweight the strength of an observation and underweight its weight. Griffin and Tversky 1992 ; Holt and Smith 2009 ; Grether 1992. This is connected with the neglect of base rates (subjects overweigh the likelihood relative to the base rates/objective priors), and the resulting « law of small numbers » (over-generalizing from small number of observations), see Rabin 2002.
* ambiguous information is taken to be confirmation of current hypothesis, which is an obstacle to learning
- non-optimal acquisition of information/memory: recency bias, search for confirming evidence rather than disconfirmatory evidence (this last category is in my opinion not that relevant for your case)
prediction: people have difficulties performing contingent reasoning on future events (Charness and Levin 2009)

Additionally, Bayes rules makes no prediction about how learners should react to zero probability events, nor how learners should structure the hypothesis space when there are no objectively known base rates of events ad nlearners have to learn about the whole structure of the underlying environment, not just a few parameters (basically, in most natural environments, such as in markets). This is the situation explored by Ortoleva and that is rarely examined in experimental settings

These departures are clues to the heuristics humans use in their judgments, which overall often approximate bayesian inference but are not equivalent. 
Although many of these deviations imply that learning should be slower than predicted by Bayes rule,  few experiments measure the rate of learning and how far it departs from Bayesian inference. We find that it does (slower). Furthermore, we also find that the quality of the inferences at each time space and across players varies a lot, following a "punctuated equilibrium" pattern. These observations are thus additional clues about the cognitive processes at play, with fine-grained information about how subjects update their beliefs in response to new information and the success and failure of their bets. Below were review some of the promising cognitive mechanisms posited to explain some of the deviations above and which can be evaluated in light of our data.

2) Cognitive mechanisms 
Interesting mechanisms have to do with how people organize the hypothesis space and sample from it. 
- sampling hypothesis
- change of paradigm when observing "surprising events" (Ortoleva)

the satisficing principle may also mediate the learning process because computations such as bayes rules are costly.