\section{Background}
In general, when sentient beings (humans or other animals) search and find solutions to their problems, this brings rewards.  At times, the solution is its own reward \cite{hedonism}, but more often the reward is survival \cite{smith_optimization_1978}, which implies finding resources directly (foraging) or indirectly (through monetary means). To strive, food search shall be economic, i.e., the search cost shall not be larger than the energetic intake \cite{pyke_optimal_1984} and optimization shall occur through either ``time minimization'' (time spent feeding is traded off for time spent on other activities) or ``energy maximization'' (fixed time in which to feed during which the animal aims to maximize its energy gain) \cite{schoener_theory_1971}.\\

The evolution of foraging behavior is subject to functional (e.g., morphologies and physical properties \cite{pyke_optimal_1984}) and environmental constraints (e.g., distribution and accessibility of food patches on complex environmental landscapes \cite{}). The capacity to sense, store and process foraging information is critical \cite{kamil_learning_1982}. Animals may obtain information through either direct experience, observation of others \cite{weigl_observational_1980}, or through basic or advanced collective intelligence mechanisms \cite{}. Foraging strategies are also assumed to evolve more rapidly than the rate at which relevant conditions change \cite{pyke_optimal_1984}. In that sense, animals are expected to adapt and optimize their foraging behaviors \cite{pyke_optimal_1977}.\\

%{\it If the animal is alone in a uniform environment, no difficulty arises.
%But if we allow for competition and for a changing environment, several
%choices of optimization procedure are possible. For example, three 
%possibilities arise if we allow just for competition:} $\rightarrow$ capacity 
%to anticipate, build relevant scenarios </Maynard1984> 

Foraging in natural environments often entails finding sparsely distributed food sites. \cite{levandowsky_swimming_1988} have suggested that microorganisms may search for sites following a Levy flight process, defined as 

\be
eq~Levyflight
\ee 

A Levy distribution is advantageous  when target sites are sparsely and randomly distributed, because the probability of returning to a previously visited site is smaller than for a Gaussian distribution. The Levy flight may also help overcome the problem of multiple animals searching for food at the same place and time \cite{Shlesinger6, viswanathan_levy_1996}. Aside from microorganisms, evidence for L\'evy flight search patterns has been identified for animals such as albatrosses \cite{viswanathan_levy_1996}, bumblebees \cite{edwards_revisiting_2007}, deer \cite{edwards_revisiting_2007}, marine predators \cite{sims_scaling_2008,humphries_environmental_2010}, spider monkeys \cite{ramos-fernandez_levy_2004} and for humans Dobe Ju/�hoansi hunter gatherers  \cite{brown_levy_2007}. {\bf [quote : Dobe Ju/�hoansi foraging hunters and foragers living in and around the Kalahari Desert in Botswana and Namibia. They have been intensively studied with special attention to their subsistence system and economy (Lee, 1979, 1993; Lee and DeVore, 1976).]} \\

Additional considerations were proposed regarding optimization of the Levy flight process, and it was shown that $\mu = 2$ is the parameter that optimizes Levy flight search patterns \cite{viswanathan_optimizing_1999}. (add other references) \\

The L\'evy flight search process has enjoyed much attention for its simplicity. Yet, more careful attention to existing data and increased sensing capacities show that Levy flights may not be the best models for some animal search strategies. It seems that the search strategies of albatrosses, bees, bumblebees and deer \cite{reynolds_displaced_2007}, that were first theorized as L\'evy flights, are in fact more complex \cite{}. It is debated whether animals and humans perform search according to a general mechanism that is hard coded in our minds, or if they switch regimes based on strategic choices and on information they can gather. If for animals with simple brain architectures and those without brains entirely, such as all micoorganisms, a pure random search mechanim (i.e., without integration of information) may be assumed, animals with highly complex brain architectures can certainly integrate and memorize more information, as well as create patterned hypotheses which in turn is critical for the optimization of search. Such considerations has led researchers to propose more advanced models, which stem from theoretical considerations and which fit the data better. {\bf [add the fancy model from Royal Society Interfaces \cite{zhao_optimal_2015}]}. Such models include the Levy-modulated correlated random walks (LMCRWs), which implies a correlated directional component (i.e., the tendency to search in a more or loss linear way, without making random turns) \cite{bartumeus_animal_2005}. Plank and James \cite{plank_optimal_2008} proposed that 2 brownian motions with regime switching between intensive and extensive search, may be indistinguishable from a Levy flight from the data, and appears to be a more efficient search strategy in the case of non-destructive foraging�. This results highlights that simple decision rules, which arguably cost little brain power such as switching between a few descrete strategies, may significantly optimize search.\\

%We study the territory covered by N Levy flights by calculating the mean number of distinct sites, $^SN(n)$, visited after n time steps on a d-dimensional, $d>2$, lattice \cite{Berkolaiko1997}
\subsection{Modern human mobility \& memory}
Human mobility in modern times is less a story of foraging than a question regarding the modern capitalist subsinstence mode.  Movement here may largely be devoted to sociality (e.g., getting to school, to work, socially interact and exchange knowledge, or for mating \cite{liljeros_web_2001}) . While early studies have relied on banknote circulation \cite{brockmann_scaling_2006} to account for mobility, the mobility question has been greatly facilitated with cellphone data \cite{gonzalez_understanding_2008}. Most datasets exhibit Levy flight patterns suggesting a universal law of foraging, search and mobility, including in virtual worlds \cite{szell_understanding_2012}.\\

It (obviously) appears that mobility patterns are not completely random. On the contrary, they encompass a significant amount of memory (people tend to live at the same place, work or study at the same place, meet socially at the same location). Song et al. \cite{song_modelling_2010} proposed a mechanism inspired from proportional growth \cite{simon_class_1955,maillart_empirical_2008} (a.k.a. preferential attachment \cite{barabasi_emergence_1999}) and incorporated it into a continuous time random walk model \cite{montroll_random_1965}, which also accounts for waiting times between displacements, which happen to be also most often heavy-tailed \cite{barabasi_origin_2005} and could be explained by task prioritization \cite{maillart_quantification_2011}, periodicity such as circadian rythms \cite{malmgren_poissonian_2008,malmgren_universality_2009} (on the contrary to travel times for animals or humans walking over large distances ?). Yet, proportional growth appears to be unsuited, or at best an incomplete model. Indeed, proportional growth prescribes that the rate at which people return to a site should grow exponentially, which appears to be counter intuitive (e.g. people stay at their home mostly at the constant rate, arriving in the evening and leaving in the morning). The other problem, which has been pointed out \cite{malmgren_universality_2009,szell_understanding_2012} is recency: in human mobility individuals tend to return more often to previously visited sites, with a memory function (which remains to be characterized).\\

%$\rightarrow$ In the physical space, there is usually an implicit %dependence between distance and time, which has by the way led to %issues regarding the validty of Levy walks \cite{Viswanathan2007} 

%<Ramos-Fernandez 2004>
%First, the length of a trajectory�s constituent steps,
%or continuous moves in the same direction, is best
%described by a power-law distribution in which the
%frequency of ever larger steps decreases as a negative
%power function of their length. The rate of this decrease is
%very close to that predicted by a previous analytical Levy walk model to be an optimal strategy to search for scarce
%resources distributed at random. Second, the frequency
%distribution of the duration of stops or waiting times also
%approximates to a power-law function.
%</Ramos-Fernandez>

\subsection{Cognitive mechanisms involved in mental search and optimization}
There are at least two reasons to consider that cognitive mechanisms involving mental search may as well exhibit L\'evy flight patterns (yet not necessarily random search as shown above). \\

First, it may be an evolutionary trait inherited from former hunter-gatherer food search strategies, which wouldn't be the only such retained trait. We have retained many other traits from this period of human development, such as .... \cite{}\\

Second, humans spend a considerable amount of their time searching for solutions to optimize the way that they gather their most needed resources (i.e., food, financial resources, pleasure, social ties, etc). Moreover, the capacity to find solutions, which are not obvious to others, may bring a substantial competitive advantage (e.g., a trader whose job is to find and implement arbitrage strategies, an entrepreneur who has found a readily implementable solution to an outstanding problem). For that, they must allocate a substantial amount of their energy resources to tasks involving cognition \cite{}, and they may in part rely on chance, which is a hallmark of high risk high return strategies \cite{}.\\

Evidence suggesting that Levy-flight search patterns occur in human cognition exists. \cite{rhodes_human_2007} performed a simple experiment involving memory retrieval. They found that retrieval occurs in bursts and these bursts often involve clusters of closely related words \cite{bousfield_analysis_1944}. {\bf [more details needed here. Read again the paper.]}\\

Radicchi et al. \cite{radicchi_rationality_2012} found that individual betting in lowest unique bid online auctions (the bidder with the lowest unique bid wins) , adopt L\'evy flight search strategies in their exploration of the ``bid space''. Unique bid online auctions prescribe that the winner should find a solution, which has not been found by any other fellow bidder (here, a justification of large excursions shall be more justified by the search of unicity equiv. to searching a food site that no one else has found yet [$\rightarrow$ rephrase and c.f., literature for that]. In the special case of lowest unique bid, they designed a model, which shows that individuals have interest to bid with the optimal strategy (power law exponent $\gamma \simeq 1.27$) only if all other players play rationally. Radicchi et al. \cite{radicchi_evolution_2012} extended their research and built a model for power law exponent optimization, based on a Moran process ({\bf ``at the end of each game, the winner of the auction generates an offspring to which her/his search exponent $\alpha$ is transmitted. The new individual enters the population endowed with an exponent $\alpha + \xi$ (with $\xi$ random mutation), while a randomly extracted individual is removed in order to maintain the population size constant"}). They found that indeed the L\'evy flight power law exponent is optimized towards a value close to 1.5. Thus, from animal and human mobility from food search to human memory to most contemporary online auction systems, L\'evy flights ermerge as a unifying concept across generations and thus, suggests that evolutionary older parts of the brain may be involved in resource retrieval and search processes \cite{baronchelli_levy_2013}.

\subsection{Research on mental search in cognitive science}

Some recent empirical work in cognitive science has exposed individuals to high incentives or confronted them with rather simple tasks.  In these contexts, were the cognitive costs of updating are low or the benefits are high, these researchers found that experimental participants update their beliefs based on new information so that the average belief sequence looks as though it had been updated using Bayes rule \cite{Griffiths2008}. Yet, at the individual level, results exhibit large variations as well as systematic departures from the Bayes rule updating mechanism:

\begin{itemize}
	\item ``biases'' in their priors (i.e., hypotheses) -- in particular deterministic biases, deviations in the updating process such as availability bias (the overestimation of the frequency of a recently observed or highly unusual event or of an event that elicits strong emotions, such as a terrorist attack) and the tendency to overestimate the magnitude of an observed quantity and underestimate its frequency in the distribution \cite{griffin1992weighing,holt2009update,grether1992testing}. Experimental participants often also over-generalize from a small number of observations \cite{rabin2002perspective}. Ambiguous information is taken to confirm a current hypotheses, a bias known as confirmation bias, which means that uncertainty in one's beliefs is reduced without sufficient cause. 
  \item {\bf sub-optimal acquisition of information/memory :} recency bias, search for confirming evidence rather than disconfirmatory evidence (this last category is in my opinion not that relevant for your case) prediction: people have difficulties performing contingent reasoning on future events (Charness and Levin 2009 \cite{charness2009origin}). 
	  
	  Additionally to problems with human biases, there are problems with updating theories; Bayes rule is no guide in the case that a learner observes an event that her prior has assigned zero probability to.  Nor is it clear how learners should structure the hypothesis space when there are no objectively known base rates of events and learners have to learn about the whole structure of the underlying environment, not just a few parameters (basically, in most natural environments, such as in markets). This is the situation explored by Ortoleva \cite{ortoleva2012modeling}, a rare examination of important aspects of realistic settings brought into an experimental environment. 
\end{itemize}

These insufficiences of Bayes rule as a model for human belief updating are clues for the heuristics humans use in their judgments and weighing of evidence. These heuristics often, but not always, approximate bayesian inference.

Although heuristic updating should imply that humans learn slower than predicted by Bayes rule, {\bf few experiments measure the rate of learning and how far--in learning spead--it departs from Bayesian inference.} \\

We find that it does: it is an order of magnitude slower. Furthermore, we also find that the quality of the inferences at each point in the time-space plain and across players varies a lot, following a ``punctuated equilibrium'' pattern: long periods of relative stasis followed by sudden shifts in belief, which then settle into a new stasis. \\


2) Cognitive mechanisms  Interesting mechanisms have to do with how people organize the hypothesis space and sample from it.  - sampling hypothesis - change of paradigm when observing ``surprising events'' (Ortoleva) \cite{ortoleva2012modeling}. The satisficing principle may also mediate the learning process because computations such as Bayes rules are perhaps more costly than reasonably well performing heuristics.\\


%{\bf [this is paragraph is not clear yet to me]} Solutions to problems, regarding estimation of complex structures in high-dimensional joint distributions can be classified into equivalence classes, which constitute the sets of structural and parametric solutions that factor into the same joint distribution \cite{pearl2009causality, Pearl2009CMR, Koller2009PGM}.  Out of these equivalence classes, it is the objective to find the equivalence class of the data generation process which has produced the observations. Note also that if there exists a {\it causal explanation} of the process, this explanation has the simplest structure in its equivalence class \cite{Koller2009PGM} and this simplest structure is unique.  Since this process -- even the structurally simplest member of its equivalence class -- might be rather complex, it may be approached over time through iterative parametric refinements of a mental model of this process, as well as through sudden structural epiphanies. Yet getting close to this solution is hard because interdependencies often have unintuitive observational consequences, or rather, observations often result from  probabilistic influence that is unintuitive to humans. Human intuitive failure with respect to probabilistic influence and its logical consequences is best illustrated through the famous Monty Hall problem \cite{Blackburn08Phil, Honderich05Phil, Upton14Statistics, Colman08Psych}. \\


When the objective is to estimate a high dimensional joint distribution on hand of lower dimensional, but complex structures, there is no unique solution, but the solution space is partitioned into classes of solutions known as equivalence classes. These equivalence classes are those sets of structural and parametric solutions that factor into the same joint distribution \cite{pearl2009causality, Pearl2009CMR, Koller2009PGM}. From an observational perspective, these lower dimensional models are thus equivalent. In other words, the processes that belong to the same equivalence class give, statistically, rise to the same observations.  Out of these equivalence classes, it is the objective to find the unique equivalence class to which the data generation process--that processes that has produced the observations--belongs. Note also that if there exists a {\it causal explanation} of the process, this explanation has the simplest structure in its equivalence class \cite{Koller2009PGM} and this simplest structure is unique. Cognitive scientists have postulated that the human concept of causation is a hard wired cognitive abstraction, used to explain observations by the simplest equivalent mechanism.  Since this process--even the structurally simplest member of its equivalence class--might be rather complex, it may be approached over time through iterative parametric refinements of a mental model of this process, as well as through sudden structural epiphanies. Yet getting close to this solution is hard because interdependencies often have unintuitive observational consequences, or rather, observations often result from  probabilistic influence that is unintuitive to humans. Human intuitive failure with respect to probabilistic influence and its logical consequences is best illustrated through the famous Monty Hall problem \cite{Blackburn08Phil, Honderich05Phil, Upton14Statistics, Colman08Psych}. \\





